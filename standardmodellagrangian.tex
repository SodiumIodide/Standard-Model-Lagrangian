\documentclass[12pt,a4paper,pagesize=pdftex]{scrartcl}
\usepackage[latin1]{inputenc}
\usepackage[english]{babel}
\usepackage{amsmath}
\usepackage{amsfonts}
\usepackage{amssymb}
\usepackage{makeidx}
\usepackage{lmodern}
\usepackage{kpfonts}
\usepackage[version=4]{mhchem}
\usepackage{color}
\usepackage{graphicx}
\usepackage[space]{grffile}
\usepackage{capt-of}
\usepackage{caption}
\usepackage{gensymb}
\usepackage{listings}
\usepackage{units}
\usepackage{hyperref}
\usepackage{float}
\lstset{frame = single, breaklines=true, postbreak=\raisebox{0ex}[0ex][0ex]
{\ensuremath{\color{red}\hookrightarrow\space}}}
\usepackage[pass,letterpaper]{geometry}
\usepackage[section]{placeins}
\graphicspath{ {.} }
\renewcommand{\thesubsection}{\thesection\alph{subsection}}
\title{Standard Model Lagrangian}
\subtitle{Including Neutrino Mass Terms}
\author{Corey Skinner}
\date{}
\linespread{1.25}

\newcommand{\icol}[1]{% inline column vector
  \left(\begin{smallmatrix}#1\end{smallmatrix}\right)%
}

\begin{document}

\pdfpageheight 11in
\pdfpagewidth 8.5in

\maketitle

% BEGIN DOCUMENT
\begin{itemize}
	\item From \textit{An Introduction to the Standard Model of Particle Physics, 2nd Edition}, W.N. Cottingham and D.A. Greenwood, Cambridge University Press, Cambridge, 2007
\end{itemize}

\begin{align*}
	\mathcal{L} = &-\frac{1}{4} B_{\mu\nu}B^{\mu\nu} - \frac{1}{8}tr\left(\mathbf{W}_{\mu\nu}\mathbf{W}^{\mu\nu}\right) - \frac{1}{2}tr\left(\mathbf{G}_{\mu\nu}\mathbf{G}^{\mu\nu}\right) && \text{(U(1), SU(2), and SU(3) gauge terms)} \\
	&+ \left(\bar{\nu}_L,\bar{e}_L\right)\tilde{\sigma}^\mu iD_\mu\icol{\nu_L\\e_L} + \bar{e}_R\sigma^\mu iD_\mu e_R + \bar{\nu}_R\sigma^\mu iD_\mu\nu_R + \text{(h.c.)} && \text{(lepton dynamical term)} \\
	&-\frac{\sqrt{2}}{\nu}\left[\left(\bar{\nu}_L,\bar{e}_L\right)\phi M^ee_R+\bar{e}_R\bar{M}^e\bar{\phi}\icol{\nu_L\\e_L}\right] && \text{(electron, muon, tauon mass term)} \\
	&-\frac{\sqrt{2}}{\nu}\left[\left(-\bar{e}_L,\bar{\nu}_L\right)\phi^{*}M^\nu\nu_R+\bar{\nu}_R\bar{M}^\nu\phi^T\icol{-e_L\\\nu_L}\right] && \text{(neutrino mass term)} \\
	& +\left(\bar{u}_L, \bar{d}_L\right)\tilde{\sigma}^\mu iD_\mu \icol{u_L\\d_L} + \bar{u}_R\sigma^\mu iD_\mu u_R + \bar{d}_R\sigma^\mu iD_\mu d_R + \text{(h.c.)} && \text{(quark dynamical term)} \\
	&- \frac{\sqrt{2}}{\nu}\left[\left(\bar{u}_L,\bar{d}_L\right)\phi M^dd_R+\bar{d}_R\bar{M}^d\bar{\phi}\icol{u_L\\d_L}\right] && \text{(down, strange, bottom mass term)} \\
	&-\frac{\sqrt{2}}{\nu}\left[\left(-\bar{d}_L,\bar{u}_L\right)\phi^{*}M^uu_R+\bar{u}_R\bar{M}^u\phi^T\icol{-d_L\\u_L}\right] && \text{(up, charmed, top mass term)} \\
	&+\overline{\left(D_\mu\phi\right)}D^\mu\phi-\frac{m_h^2\left[\bar{\phi}\phi-\frac{\nu^2}{2}\right]^2}{2\nu^2} && \text{(Higgs dynamical and mass term)}
\end{align*}

Where (h.c.) means Hermitian conjugate of preceding terms, $\bar{\phi}=\text{(h.c.)}\phi=\phi^\dagger=\phi^{*T}$, and the derivative operators are:

\begin{equation*}
	D_\mu\icol{\nu_L\\e_L}=\left[\partial_\mu-\frac{ig_1}{2}B_\mu+\frac{ig_2}{2}\mathbf{W}_\mu\right]\icol{\nu_L\\e_L}
\end{equation*}

\begin{equation*}
	D_\mu\icol{u_L\\d_L}=\left[\partial_\mu-\frac{ig_1}{6}B_\mu+\frac{ig_2}{2}\mathbf{W}_\mu+ig\mathbf{G}_\mu\right]\icol{u_L\\d_L}
\end{equation*}

\begin{equation*}
	D_\mu\nu_R=\partial_\mu\nu_R
\end{equation*}

\begin{equation*}
	D_\mu e_R=\left[\partial_\mu-ig_1B_\mu\right]e_R
\end{equation*}

\begin{equation*}
	D_\mu u_R=\left[\partial_\mu+\frac{i2g_1}{3}B_\mu+ig\mathbf{G}_\mu\right]u_r
\end{equation*}

\begin{equation*}
	D_\mu d_R=\left[\partial_\mu - \frac{ig_1}{3}B_\mu+ig\mathbf{G}_\mu\right]d_R
\end{equation*}

\begin{equation*}
	D_\mu\phi=\left[\partial_\mu+\frac{ig_1}{2}B_\mu+\frac{ig_2}{2}\mathbf{W}_\mu\right]\phi
\end{equation*}

$\phi$ is a 2-component comples Higgs field. Since $\mathcal{L}$ is SU(2) gauge invariant, a gauge can be chosen so $\phi$ has the form:

\begin{equation*}
	\phi^T=\frac{\left(0,\nu+h\right)}{\sqrt{2}}
\end{equation*}

\begin{equation*}
	<\phi>_0^T=\text{(expectation value of}\phi)=\frac{\left(0,\nu\right)}{\sqrt{2}}
\end{equation*}

Where $\nu$ is a real constant such that $\mathcal{L}_\phi=\overline{\left(\partial_\mu\phi\right)}\partial^\mu\phi-\frac{\mu\phi-m_h^2\left[\bar{\phi}\phi-\frac{\nu^2}{2}\right]^2}{2\nu^2}$ is minimized, and $h$ is a residual Higgs field. $B_\mu$, $\mathbf{W}_\mu$, and $\mathbf{G}_\mu$ are the gauge boson vector potentials, and $\mathbf{W}_\mu$ and $\mathbf{G}_\mu$ are composed of $2\times3$ and $3\times3$ traceless Hermitian matrices. Their associated field tensors are:

\begin{equation*}
	B_{\mu\nu}=\partial_\mu B_\nu-\partial_\nu B_\mu
\end{equation*}

\begin{equation*}
	\mathbf{W}_{\mu\nu}=\partial_\mu\mathbf{W}_\nu-\partial_\nu\mathbf{W}_\mu+ig_2\frac{\left(\mathbf{W}_\mu\mathbf{W}_\nu-\mathbf{W}_\nu\mathbf{W}_\mu\right)}{2}
\end{equation*}

\begin{equation*}
	\mathbf{G}_{\mu\nu}=\partial_\mu\mathbf{G}_\nu-\partial_\nu\mathbf{G}_\mu+\left(\mathbf{G}_\mu\mathbf{G}_\nu-\mathbf{G}_\nu\mathbf{G}_\mu\right)
\end{equation*}

The non-matrix $A_\mu$, $Z_\mu$, $W_\mu^\pm$ bosons are mixtures of $\mathbf{W}_\mu$ and $B_\mu$ components, according to the weak mixing angle $\theta_w$:

\begin{equation*}
	A_\mu=W_{11\mu}\sin\left(\theta_w\right)+B_\mu\cos\left(\theta_w\right)
\end{equation*}

\begin{equation*}
	Z_\mu=W_{11\mu}\cos\left(\theta_w\right)-B_\mu\sin\left(\theta_w\right)
\end{equation*}

\begin{equation*}
	W_\mu^+=W_\mu^{-*}=\frac{W_{12\mu}}{\sqrt{2}}
\end{equation*}

\begin{equation*}
	B_\mu=A_\mu\cos\left(\theta_w\right)-Z_\mu\sin\left(\theta_w\right)
\end{equation*}

\begin{equation*}
	W_{11\mu}=-W_{22\mu}=A_\mu\sin\left(\theta_w\right)+Z_\mu\cos\left(\theta_w\right)
\end{equation*}

\begin{equation*}
	W_{12\mu}=W_{21\mu}^*=\sqrt{2}W_\mu^+
\end{equation*}

\begin{equation*}
	\sin^2\left(\theta_w\right)=.2325(4)
\end{equation*}

The fermions include the leptons $e_R$, $e_L$, $\nu_R$, $\nu_L$ and quarks $u_R$, $u_L$, $d_R$, $d_L$. They all have implicit 3-component generation indices, $e_i=\left(e,\mu,\tau\right)$, $\nu_i=\left(\nu_e,\nu_\mu,\nu_\tau\right)$, $u_i=\left(u,c,t\right)$, $d_i=\left(d,s,b\right)$, which contract into the fermion mass matrices $M_{ij}^e,M_{ij}^\nu,M_{ij}^u,M_{ij}^d$, and implicit 2-component indices which contract into the Pauli matrices:

\begin{equation*}
	\sigma^\mu=\left[\left(\begin{matrix}1&0\\0&1\end{matrix}\right),\left(\begin{matrix}0&1\\1&0\end{matrix}\right),\left(\begin{matrix}0&-i\\i&0\end{matrix}\right),\left(\begin{matrix}1&0\\0&-1\end{matrix}\right)\right]
\end{equation*}

\begin{equation*}
	\tilde{\sigma}^\mu=\left[\sigma^0,-\sigma^1,-\sigma^2,-\sigma^3\right]
\end{equation*}

\begin{equation*}
	tr\left(\sigma^i\right)=0
\end{equation*}

\begin{equation*}
	\sigma^{\mu\dagger}=\sigma^\mu
\end{equation*}

\begin{equation*}
	tr\left(\sigma^\mu\sigma^\nu\right)=2\delta^{\mu\nu}
\end{equation*}

The quarks also have implicit 3-component color indices which contract into $\mathbf{G}_\mu$. So $\mathcal{L}$ really has implicit sums over 3-component generation indices, 2-component Pauli indices, 3-component color indices in the quark terms, and 2-component SU(2) indices in $\left(\bar{\nu}_L,\bar{e}_L\right),\left(\bar{u}_L,\bar{d}_L\right),\left(-\bar{e}_L,\bar{\nu}_L\right),\left(-\bar{d}_L,\bar{u}_L\right),\bar{\phi},\mathbf{W}_\mu,\icol{\nu_L\\e_L},\icol{u_L\\d_L},\icol{-e_L\\\nu_L},\icol{-d_L\\u_L},\phi$.

The electroweak and strong coupling constants, Higgs vacuum expectation value (VEV), and Higgs mass are:

\begin{equation*}
	g_1=\frac{e}{\cos\left(\theta_w\right)}
\end{equation*}

\begin{equation*}
	g_2=\frac{e}{\sin\left(\theta_w\right)}
\end{equation*}

\begin{equation*}
	g>6.5e=g\left(m_\tau^2\right)
\end{equation*}

\begin{equation*}
	\nu=246\text{ GeV (PDG)} \approx \sqrt{2}\cdot180\text{ GeV (CG)}
\end{equation*}

\begin{equation*}
	m_h=125.02(30)\text{ GeV}
\end{equation*}

Where $e=\sqrt{4\pi\alpha\hbar c}=\sqrt{\frac{4\pi}{137}}$ in natural units. Rewriting some things provides the mass of $A_\mu$, $Z_\mu$, $W_\mu^\pm$:

\begin{equation*}
	-\frac{1}{4}B_{\mu\nu}B^{\mu\nu}-\frac{1}{8}tr\left(\mathbf{W}_{\mu\nu}\mathbf{W}^{\mu\nu}\right)=-\frac{1}{4}A_{\mu\nu}A^{\mu\nu}-\frac{1}{4}Z_{\mu\nu}Z^{\mu\nu}-\frac{1}{2}\mathcal{W}_{\mu\nu}^-\mathcal{W}^{+\mu\nu}+\text{(higher order terms)}
\end{equation*}

\begin{equation*}
	A_{\mu\nu}=\partial_\mu A_\nu-\partial_\nu A_\mu
\end{equation*}

\begin{equation*}
	Z_{\mu\nu}=\partial_\mu Z_\nu-\partial_\nu Z_\mu
\end{equation*}

\begin{equation*}
	W_{\mu\nu}^\pm=D_\mu W\nu^\pm - D_\nu W_\mu^\pm
\end{equation*}

\begin{equation*}
	D_\mu W_\nu^\pm = \left[\partial_\mu \pm ieA_\mu\right] W_\nu^\pm
\end{equation*}

\begin{equation*}
	D_\mu<\phi>_0=\frac{i\nu}{\sqrt{2}}\icol{\frac{g_2W_{12\mu}}{2}\\ \frac{g_1B_\mu}{2}+\frac{g_2W_{22\mu}}{2}}=\frac{ig_2\nu}{2}\icol{\frac{W_{12\mu}}{\sqrt{2}}\\ \frac{\left(B_\mu\sin\left(\theta_w\right)/\cos\left(\theta_w\right)+W_{22\mu}\right)}{\sqrt{2}}} = \frac{ig_2\nu}{2}\icol{W_\mu^+\\\frac{-Z_\mu}{\sqrt{2}\cos\left(\theta_w\right)}}
\end{equation*}

\begin{equation*}
	m_A=0
\end{equation*}

\begin{equation*}
	m_{W^\pm}=\frac{g_2\nu}{2}=80.425(38)\text{ GeV}
\end{equation*}

\begin{equation*}
	m_Z=\frac{g_2\nu}{2\cos\left(\theta_w\right)}=91.1876(21)\text{ GeV}
\end{equation*}

Ordinary 4-component Dirac fermions are composed of the left and right handed 2-component fields:

\begin{align*}
	&e=\icol{e_{L1}\\e_{R1}},\nu_e=\icol{\nu_{L1}\\\nu_{R1}}, u=\icol{u_{L1}\\u_{R1}},d=\icol{d_{L1}\\d_{R1}},&&\text{ (electron, electron neutrino, up and down quark)}\\
	&\mu=\icol{e_{L2}\\e_{R2}},\nu_\mu=\icol{\nu_{L2}\\\nu_{R2}},c=\icol{u_{L2}\\u_{R2}},s=\icol{d_{L2}\\d_{R2}},&&\text{ (muon, muon neutrino, charm, and strange quark)}\\
	&\tau=\icol{e_{L3}\\e_{R3}},\nu_\tau=\icol{\nu_{L3}\\\nu_{R3}},t=\icol{u_{L3}\\u_{R3}},b=\icol{d_{L3}\\d_{R3}},&&\text{ (tauon, tauon neutrino, top and bottom quark)}\\
	&\gamma^\mu=\left(\begin{smallmatrix}0&\sigma^\mu\\\tilde{\sigma}^\mu&0\end{smallmatrix}\right)\text{ where }\gamma^\mu\gamma^\nu+\gamma^\nu\gamma^\mu=2Ig^{\mu\nu}&&\text{ (Dirac gamma matrices in chiral representation)}
\end{align*}

The corresponding antiparticles are related to the particles according to $\psi^e=-i\gamma^2\psi^*$ or $\psi_L^e=-i\sigma^2\psi^*_R$, $\psi_R^e=i\sigma^2\psi_L^*$. The fermion charges are the coefficients of $A_\mu$ when (8,10) are substituted into either the left or right handed derivative operators (2-4). The fermion masses are the singular values of the $3\times3$ fermion mass matrices $M^\nu, M^e, M^u, M^d$:

\begin{equation*}
	M^e=\mathbf{U}_L^{e\dagger}\left(\begin{matrix}m_e&0&0\\0&m_\mu&0\\0&0&m_\tau\end{matrix}\right)\mathbf{U}_R^e
\end{equation*}

\begin{equation*}
	M^\nu=\mathbf{U}_L^{\nu\dagger}\left(\begin{matrix}m_{\nu_e}&0&0\\0&m_{\nu_\mu}&0\\0&0&m_{\nu_\tau}\end{matrix}\right)\mathbf{U}_R^\nu
\end{equation*}

\begin{equation*}
	M^u=\mathbf{U}_L^{u\dagger}\left(\begin{matrix}m_u&0&0\\0&m_c&0\\0&0&m_t\end{matrix}\right)\mathbf{U}_R^u
\end{equation*}

\begin{equation*}
	M^d=\mathbf{U}_L^{d\dagger}\left(\begin{matrix}m_d&0&0\\0&m_s&0\\0&0&m_b\end{matrix}\right)\mathbf{U}_R^d
\end{equation*}

\begin{align*}
	&m_e=.510998910(13)\text{ MeV}, &&m_{\nu_e}\sim.001-2\text{ eV}, &&&m_u=1.7-3.1\text{ MeV}, &&&&m_d=4.1-5.7\text{ MeV} \\
	&m_\mu=105.658367(4)\text{ MeV}, &&m_{\nu_\mu}\sim.001-2\text{ eV}, &&&m_c=1.18-1.34\text{ GeV}, &&&&m_s=80-130\text{ MeV} \\
	&m_\tau=1776.84(17)\text{ MeV}, &&m_{\nu_\tau}\sim.001-2\text{ eV}, &&&m_t=171.4-174.4\text{ GeV}, &&&&m_b=4.13-4.37\text{ GeV}
\end{align*}

Where the $\mathbf{U}$s are $3\times3$ unitary matrices $\left(\mathbf{U}^{-1}=\mathbf{U}^\dagger\right)$. Consequently the ``true fermions'' with definite masses are actually linear combinations of those in $\mathcal{L}$, or conversely the fermions in $\mathcal{L}$ are linear combinations of the true fermions:

\begin{equation*}
	e'_L=\mathbf{U}_L^ee_L\text{ , }e'_R=\mathbf{U}_R^ee_R\text{ , }\nu'_L=\mathbf{U}_L^\nu\nu_L\text{ , }\nu'_R=\mathbf{U}_R^\nu\nu_R\text{ , }u'_L=\mathbf{U}_L^uu_L\text{ , }u'_R=\mathbf{U}_R^uu_R\text{ , }d'_L=\mathbf{U}_L^dd_L\text{ , }d'_R=\mathbf{U}_R^dd_R
\end{equation*}

\begin{equation*}
	e_L=\mathbf{U}_L^{e\dagger}e'_L\text{ , }e_R=\mathbf{U}_R^{e\dagger}e'_R\text{ , }\nu_L=\mathbf{U}_L^{\nu\dagger}\nu'_L\text{ , }\nu_R=\mathbf{U}_R^{\nu\dagger}\nu'_R\text{ , }u_L=\mathbf{U}_L^{u\dagger}u'_L\text{ , }u_R=\mathbf{U}_R^{u\dagger}u'_R\text{ , }d_L=\mathbf{U}_L^{d\dagger}d'_L\text{ , }d_R=\mathbf{U}_R^{d\dagger}d'_R
\end{equation*}

When $\mathcal{L}$ is written in terms of the true fermions, the $\mathbf{U}$s fall out except in $\bar{u}'_L\mathbf{U}_L^u\tilde{\sigma}^\mu W_\mu^\pm\mathbf{U}_L^{d\dagger}d'_L$ and $\bar{\nu}'_L\mathbf{U}_L^\nu\tilde{sigma}^\mu W_\mu^\pm\mathbf{U}_L^{e\dagger}e'_L$. Because of this, and some absorption of constants into the fermion fields, all the parameters in the $\mathbf{U}$s are contained in only four components of the Cabibbo-Kobayashi-Maskawa matrix $\mathbf{V}^q=\mathbf{U}_L^u\mathbf{U}_L^{d\dagger}$ and four components of the Pontecorvo-Maki-Nakagawa-Sakata matrix $\mathbf{V}^l=\mathbf{U}_L^\nu\mathbf{U}_L^{e\dagger}$. The unitary matrices $\mathbf{V}^q$ and $\mathbf{V}^l$ are often parameterized as:

\begin{equation*}
	\mathbf{V}=\left(\begin{matrix}1&0&0\\0&c_{23}&s_{23}\\0&-s_{23}&c_{23}\end{matrix}\right)\left(\begin{matrix}e^{\frac{-i\delta}{2}}&0&0\\0&1&0\\0&0&e^{\frac{i\delta}{2}}\end{matrix}\right)\left(\begin{matrix}c_{13}&0&s_{13}\\0&1&0\\-s_{13}&0&c_{13}\end{matrix}\right)\left(\begin{matrix}e^{\frac{i\delta}{2}}&0&0\\0&1&0\\0&0&e^{\frac{-i\delta}{2}}\end{matrix}\right)\left(\begin{matrix}c_{12}&s_{12}&0\\-s_{12}&c_{12}&0\\0&0&1\end{matrix}\right)
\end{equation*}

\begin{equation*}
	c_j=\sqrt{1-s_j^2}
\end{equation*}

\begin{align*}
	&\delta^q=69(4)\text{ deg, } &&s_{12}^q=0.2253(7)\text{, } &&&s_{23}^q=0.041(1)\text{, } &&&&s_{13}^q=0.0035(2),\\
	&\delta^l=?\text{, } &&s_{12}^l=0.560(16)\text{, } &&&s_{23}^l=0.7(1)\text{, } &&&&s_{13}^l=0.153(28)
\end{align*}

$\mathcal{L}$ is invariant under a U(1) $\otimes$ SU(2) gauge transformation with $U^{-1}=U^\dagger$, $detU=1$, $\theta$ real:

\begin{equation*}
	\mathbf{W}_\mu\rightarrow U\mathbf{W}_\mu U^\dagger - \left(\frac{2i}{g_2}\right)U\partial_\mu U^\dagger
\end{equation*}

\begin{equation*}
	\mathbf{W}_{\mu\nu}\rightarrow U\mathbf{W}_{\mu\nu} U^\dagger
\end{equation*}

\begin{equation*}
	B_\mu\rightarrow B_\mu + \left(\frac{2}{g_1}\right)\partial_\mu\theta
\end{equation*}

\begin{equation*}
	B_{\mu\nu}\rightarrow B_{\mu\nu}
\end{equation*}

\begin{equation*}
	\phi\rightarrow e^{-i\theta}U\phi
\end{equation*}

\begin{equation*}
	\icol{\nu_L\\e_L}\rightarrow e^{i\theta}U\icol{\nu_L\\e_L}
\end{equation*}

\begin{equation*}
	\icol{u_L\\d_L}\rightarrow e^{-\frac{i\theta}{3}}U\icol{u_L\\d_L}
\end{equation*}

\begin{align*}
	&\nu_R\rightarrow \nu_R && u_R\rightarrow e^{-\frac{4i\theta}{3}}u_R \\
	&e_R\rightarrow e^{2i\theta}e_R && d_R\rightarrow e^{\frac{2i\theta}{3}}d_R
\end{align*}

And under an SU(3) gauge transformation with $V^{-1}=V^\dagger$, $detV=1$, and:

\begin{equation*}
	\mathbf{G}_\mu\rightarrow V\mathbf{G}_\mu V^\dagger - \left(\frac{i}{g}\right)V\partial_\mu V^\dagger
\end{equation*}

\begin{equation*}
	\mathbf{G}_{\mu\nu}\rightarrow V\mathbf{G}_{\mu\nu}V^\dagger
\end{equation*}

\begin{align*}
	&u_L \rightarrow V u_L && d_L \rightarrow V d_L \\
	&u_R \rightarrow V u_R && d_R \rightarrow V d_R
\end{align*}

\end{document}
